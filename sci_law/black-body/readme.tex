\documentclass{article}

\begin{document}
\section{Purpose}

The purpose of this directory was originally is to find what the apparent color
of a black body is, which is quite distinct from its black body color
temperature defined as the wavelength of maximum intensity given by Wien's
displacement law. In particular, the apparent color is apparent at temperatures
far below the black body color temperature because the human eye cannot detect
the majority of the color in the infrared and is integrating over a small
region of the tail of the black body curve.

One might first wish to ask, what would an ideal eye would see if it had equal
sensitivities to all wavelengths of light? But an ideal eye isn't
scientifically meaningful to ask since color perception is a psychological
phenomenon and a so-called \emph{qualia}. One can instead only ask how the
colors perceived depend on the measurable spectral intensity.

\section{Color Perception} % this discussion is effectively a review of the cited conference proceeding and Wikipedia articles

The eye does not see the spectral distribution, but its convolution with the
so-called cone fundamentals, or sensitivities of the cones of the eye. That is,

$$S_i = \int f_i(\lambda) g(\lambda) \, d\lambda \text{ where } i
\in \{r, g, b\}$$

The cone fundamenals are rather broad (blue spans ~ 100 nm, green and red~ 200
nm, where span is defined as ~5\% of maximum sensitivity) and the perception of
a color spectrum beyond monochromatic colors is made possible by this. For
example, a light blue is a mix of all color spectrum (white light) with blue
light, while a dark blue is exclusively blue frequencies. The extraordinary
diversity in color perception is due to the many possible values for each
signal received, though it is insignificant relative to the total number of
possible spectral distributions (a mathematically precise definition of the
relative cardinalities is wanting). There are also psychological phenomena that
need to be accounted for in color perception, including contrast.

Only three scaled delta functions of monochromatic colors, RGB, are needed to
reproduce a large range of the human color perception. Where they do not it is
not because of the single-valuedness, since in the given model the signal
received is merely the integral, which is a single integer value just like the
intensity of a monochromatic frequency. Rather, it is because the broad range
of sensitivies of the cones, or more exactly the overlap of the sensitivities
of the cones, allows only a subset of relative intensity ratios to be obtained
when using 3 monochromatic frequencies. In particular, disregard total
brightness (which has an effect on perceived color) and consider the
ratios $G/R$ and $B/R$ where these are monochromatic frequencies.  These
ratios, even if varied independently in $\mathbb{R}^2$ to fully cover the
parameterization domain making a Cartesian grid, only cover part of the
convoluted parameterization domain, that is, $S_g/S_r$ and $S_b/S_r$. This is
shown in the triangle of linear interpolation in the $xy$ chromaticity
diagrams. In fact, the definitions of $x$ and $y$ in the chromaticity diagram
are similar to the ratios given here in ignoring the total brightness. The $x$
and $y$ are normalized intensities of components, so that the third components
fraction is given by balance as $z = 1 - x - y$. Hence the color gamut for the
3-cone human perception may be represented on two dimensions.

RGB is a color specification, not a set of three monochromatic frequencies--the
display merely has to display three colors which appear red, green, and blue,
which can be further separated from each other. Those RGB displays which have
red at a higher wavelength and blue at a lower wavelength (thus green more
separated from blue) will have a fuller color gamut than another because they
will be vary $S_r$ and $S_b$ over a greater fraction of the grid defined by
$S_r \times S_b$. It also isn't necessary the display have three monochromatic
frequencies of light: in practice, they may have spectral distributions
corresponding to the colors, but the same reasoning applies as to their
positions and shapes to their ability to reproduce a fuller color gamut, that
is, narrower and farther apart will have a fuller color gamut. From the paper \emph{Color
Spaces for Computer Graphics}, a color triangle on a chromaticity diagram was
defined only for monochromatic primary colors, though. From that same reference
there is:

\begin{enquote}
Any three wavelengths of light can be mixed in varying proportions to create
many different colors. Some sets of three wavelengths can produce more colors
than others, but the particular characteristics of the three human receptor
systems make it impossible for any such set of three primary colors to
duplicate all colors. The three primaries which can be mixed to produce the
greatest number of colors are particular wavelengths of red, green and blue.
For this reason, most color display systems are based on three light sources
which are as close to these colors as possible.
\end{enquote}

The $xy$ chromaticity region is highly non-linear, owing to the highly
non-linear convolutions with the cone fundamentals. Any finite set of
monochromatic colors will only approximate the human perception color gamut so
long as it exists inside the chromaticity region (and you can't cheat by
choosing some wavelength outside the chromaticity region like 350 nm for blue
or 800 nm for red, because the cone fundamental sensitivity at that wavelength
is too low for you to practically do that).

\subsection{Brightness and the Color Solid}

There is an important additional dimension to color perception than just the
relative intensities of the component lights, and that is the total intensity.
That additional dimension makes a color solid. If you use $R$, $G$, and $B$ as
coordinate axes you would get a color solid which accounts for all color perception
outside of psychological effects. But it is better to specify $x$, $y$, and
total brightness as the three dimensions.

Perhaps the simplest measure of brightness, which is related to lightness, is
the mean of RGB, that is, $\mu = (R + G + B)/3$. The chromaticity diagram,
which is generally drawn at a constant lightness (which is a relative
brightness).  When the lightness/brightness (related to intensity) goes to zero
necessarily all colors will appear black. As intensity increases, though, a
color does not approach white. This fact is reflected by sphere color solids 
e.g., Runge's or Munsell's color sphere, with the poles corresponding to black and
white, and the latitudes being colors. This limiting behavior also
distinguishes the HSL and HSV cylindrical models (see \S\ref{sec:sat-hue-trans}).

\subsection{Hue/Saturation Coordinate Transform}\label{sec:sat-hue-trans}

Hue and Saturation are alternative descriptions of color to $R$, $G$, $B$.
Hue and saturation are the result of a coordinate transform of the $xy$
chromaticity diagram to place the origin on the white point and describe the
coordinates in terms of $r\theta$. As before observed, the $xy$ chromaticity
diagram is a slice of the color solid of constant lightness, and a third
dimension corresponding to a cylindrical coordinate would be required to
reconstruct the color solid. But since lightness is related, however
non-linearly, to intensity (and saturation and hue are largely independent of
this), it is possible to section this and have the other two degrees of freedom
fully represented.

In the Hue/Saturation/Value coordinate transforms, the third dimension
corresponds to intensity, and vanishes to black when the intensity goes to
zero, but remains the same color when the intensity goes to infinity, as is
physically expected (c.f. figure 10a of \emph{ibid}). In an alternative
Hue/Relative Chroma/Intensity (this is the name given in the paper, but modern
nomenclature is Hue/Saturation/Lightness) color space, the perceived color goes
to white as the intensity goes to infinity (c.f. figure 8a of \emph{ibid}).

The conversion from $xy$-chromaticity diagrams (which omit the effect on color
perception of the lightness $\mu$) to cylindrical models of hue, saturation,
and lightness was due to the desire to make a more intuitive set of measures
used for mixing colors like paints. Most notably, mixing black and white paint
with various saturations of hues to make different lightness. In practice it is
not possible for lightness and saturation to be independent. Mixing in black or
white changes both lightness and saturation since saturation is before defined
as \enquote{the absence of whiteness}; one sees the utility of the Hue/Relative
Chroma/Intensity color solid in that there is a symmetry in going toward black
at one end and white at the other, which would correspond to adding black and
white paint.

The color chooser in Microsoft PowerPoint 2017 has an HSL color model Hue, Sat
(saturation), and Lum (probably luminance). A zero saturation color is gray and
a maximum saturation color is the qualitatively purest of that color in the
square with the displayed luminance they chose.  They include a third slider
for luminance which can cause any color to go from black to white--this accords
with the intuitive notion of mixing black and white paint, but of course
changing luminance without a relative chroma (which is relative to the
intensity) would not do this, and so this is an incorrect choice of
nomenclature, however, one that appears standardized in mondern terms since
Hue/Relative Chroma/Intensity is called Hue/Saturation/Lightness.

\subsection{Mathematical Properties of Spectra Related to Hue/Saturation/Lightness}

In general, I think it is advantageous to define Hue/Saturation/Value in
terms of mathematical properties of spectra, even if that definition is
quantitatively inexact. 

\begin{itemize}
\item lightness is increasing with a scaling in the intensity (I see already
that brightness is not necessarily proportional to luminance, which is the
optics name for the general intensity, but they increase).
\item saturation is increasing with a decrease in white baseline, and also an
increase in chromatic peak prominence (peak prominence is the height of the
peak relative to the nearest valley, both of which are defined as extrema in
the spectral curve).
\item hue is changing when the intensity spectrum is shifted (see mathematical
definition of shift). It isn't meaningful to speak about increasing hue, in the
same way that it isn't very meaningful to speak about increasing angle, since
the angle change is an orientation change, not a magnitude change, and both the
origin and the orientation (meaning clockwise or counterclockwise) of color
wheels is arbitrary.
\end{itemize}

Since mixing paints is a physical phenomenon, and this model was made for such
a physical phenomenon, it can be seen that these alternative descriptions are
actually more physical---adding white paint to colored paint, assuming
superposition, will only increase the baseline reflectance, while adding black
paint to colored paint, assuming superposition, will only decrease the baseline
reflectance. Both of these would generally change lightness and saturation,
though not (by definition) relative chroma.

The saturation is an important measure because most colors we see are
illuminated by sunlight, which is (approximately) white at sea level after
absorption and other phenomena such as scattering. Since the absorption of
wavelengths of a particular light are always partial, and there tends to be a
reflection of the source light, white light is always a component. Said another
way, most objects viewed have a significant non-zero baseline in addition to
peaks and troughs. Saturation increases as a baseline of white light is
subtracted and is maximum when zero, though there may be other complications.

The corresponding changes in spectra to changes in the color perception
coordinates of the HSV model are given in \texttt{illus.py}. Note, however,
that because the same colors perceived by humans can have an infinite number of
spectra, one should take this relationship as causally from changes to spectra
to changes in perception. Several different changes in the spectra can effect
the same change in perception. In the illustration a broad-band spectrum across
the whole visible range is given, but only three monochromatic frequencies are
needed to illustrate the changes in color perception that would come from
'spectral changes' (that is, just let the spectrum correspond to the
frequencies in a given RGB color model).

\subsection{Psychometric Models}

Psychometric models of color perception usually involve other measured such as
distance of a color from the white point than just its RGB value---but they
don't introduce other dimensions, because the physical input to the brain from
the eyes is complete with an RGB (or HSL, or HSV, or other 3-dimensional)
color model. There are psychological phenomena where the apparent brightness of a
color at the same physical brightness differs (colors tend to appear brighter,
even by a factor of 2, than white at the same brightness). In addition,
juxtaposition of colors can have an effect on the perception of each. Note here
brightness is not brilliance or intensity.

\subsubsection{Perceptual Uniformity}

The major update on the 1931 CIE diagram in later years was in developing color
models which corresponded to perception, including the just discussed
coordinate transform of the $xy$ chromaticity diagram. Also what was changed
was the $xy$-chromaticity diagram scaling so that displayed colors would have
perceptual uniformity, which is the perception of equal gradation everywhere. 

One application where this is relevant is in using color maps for quantitative
data, e.g., for filled contour plots or their almost-continuous equivalent heat
maps (see matplotlib's available perceptually uniform color maps).

\subsection{A note on ROY G BIV classification}

As part of the original classification of colors, indigo must have referred to
a completely saturated, monochromatic (single wavelength) light in the spectrum
from 400nm to 700nm. It was defined to symmetrize the monochromatic colors with
green at the center in Red-Orange-Yellow-Green-Blue-Indigo-Violet, but it is
difficult for humans to tell any difference between monochromatic shades of
blue, so it appears there is only blue and violet after green. Now modern
definitions of indigo may be a \enquote{dark blue}, which is a perception
achieved by mixing with black, which is actually just an absence of light or
brightness, and would maintain the monochromaticity. However, I have seen
the indigo RGB defined as $(75,0,130)$, which has a high component of
red light and clearly doesn't correspond to a monochromatic light.

\section{About Shades of Gray}

If gray is a mixture of white and black, are the shades of gray only due to
changes in baseline intensity? That is, can gray be represented just by scaling
the uniform energy spectrum? Yes, in the RGB model, grays are equal values of
$R$, $G$, and $B$ between white (8-bit convention 255) and black (8-bit
convention 0). In the HSL model, grays are at saturation 0 (for any hue) and
vary by luminance between white (8-bit convention 255) and black (8-bit
convention 0).

Of course, slightly deviant values, such as an RGB of $(100, 101, 100)$, will
still appear gray. But gray scale is defined as 1-dimensional.

\section{References}

\noindent Joblove, George~H., and Donald~Greenberg. "Color spaces for computer graphics."
\emph{Proceedings of the 5th annual conference on Computer graphics and
interactive techniques}. 1978.

\end{document}
